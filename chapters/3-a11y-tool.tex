%!TEX root = ../dissertation.tex
\begin{savequote}[75mm]
Nulla facilisi. In vel sem. Morbi id urna in diam dignissim feugiat. Proin molestie tortor eu velit. Aliquam erat volutpat. Nullam ultrices, diam tempus vulputate egestas, eros pede varius leo.
\qauthor{Quoteauthor Lastname}
\end{savequote}

\chapter{A11Y Tool}
\section{Preparation}
\subsection{Requirements}
\subsection{Diagrams}
\subsection{Design}

\section{Deliverable}
\newthought{Lorem ipsum dolor sit amet},

This chapter should describe what was actually produced: the programs which
were written, the hardware which was built, the theory which was developed or
the new scientific knowledge acquired.

For software projects, give a high-level overview of your realisation of the
design. Describe the general organisation of any body of code, web pages,
database tables, etc, that you have created. Highlight any particularly
noteworthy aspects, e.g., specialised algorithms, but avoid excessive low-level
detail. Diagrams and examples are usually valuable.

For research projects, provide a detailed overview of how the research was
executed (e.g., participants involved, etc). the research results and their
analysis. Include a description of any statistical analysis methods used.
Highlight any particularly noteworthy aspects, e.g., especially interesting
results. Graphs/charts and examples are usually essential. When reporting
statistical analysis, do not merely present the statistics without interpreting
their meaning for the reader – e.g., what are the implications of the findings?
Where applicable, based on the results and analysis, present a set of
recommendations, guidelines, or itemised list of contributions to knowledge
that may be derived from your work.

This section should be answering the question: “What did the project actually
produce?”
