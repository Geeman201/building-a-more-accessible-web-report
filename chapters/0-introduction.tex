%!TEX root = ../dissertation.tex
\chapter{Introduction}
\label{introduction}
20\% of the UK population reported they have a disability
\citep{UkGovFamilySurvey}.
That is approximately 13.3 million people. In the physical world,
companies are by law
bound \citep{DDA} to ensure that this minority are able to access their services; be it by
leaving enough room to accomodate wheel chairs; or offering large text prints
of their products. In the digital world there are no such laws and thus the web can be a
difficult place for these users to consume services and content.

A range of assistive
tools aim to improve the experience by targeting a selection of
disabilities and offering other means to consume the content. For example,
JAWS \citep{JAWS} targets visually impaired users; reading the content,
labelling actions and offering keyboard shortcuts to navigate. The
problem
with all these tools is that they rely upon Software Engineers to produce
content semantically (using HTML, CSS and Javascript) and add metadata using,
until recently,
very loose specifications \citep{WCAG} to enable the tools to better process
the content.

Although difficult to assess some companies believe \textasciitilde70\% of all
websites are inaccessible to all users \citep{Slate} \citep{SightAndSound}.
This identifies a clear gap in knowledge within the design/development community
when it comes to producing accessible web applications.

This project aims to to reduce the size of the gap by enabling the development
community through education and `easy to use' tools to think about and
implement accessibility whilst coding.

This report begins with a background chapter to discuss the topic, the business
requirement, this projects goals and the methodology that will be used to
compelete it. Chapter 2 will discuss the first deliverable an
accessibility guide. It will document the preparation that was undertaken and
then then how the deliverable was designed and implemented. Chapter 3 will be
of similar structure, using the knowledge learned building the guide it will
discuss the creation of a tool for assessing accessibilty issues. A
critical evaluation of the project will follow discussing the successes
and shortcomings as well as personal reflections on the journey. Followed by
of course a concluding paragraph, references and appendecies.

% The Introduction should clearly lay out the principal motivation for the
% project
% and briefly outline how the work fits into the broad area of surrounding
% Computer Science. The Introduction should outline the remaining structure of
% the report.
