%!TEX root = ../dissertation.tex
\begin{savequote}[75mm]
This is some random quote to start off the chapter.
\qauthor{Firstname lastname}
\end{savequote}

\chapter{Preparation}

\section{Identification of Stakeholders}
This project has a small number of stakeholders.
\begin{itemize}
  \item Users of assistive technologies, without them this whole project would be impossible and have no impact
  \item My two supervisors — Academic and Company, these are the people I will eventually present my deliverables to (as well as the
  community).
  \item Web development engineers — If I want to improve the experience for the users these are the people that need to produce better code
\end{itemize}

\section{Way of working}
Having gained five years industry experience I want to execute the project as it were for a client. For Capgemini this always involves
following a defined process to ensure speed and quality. Due to the academic nature, I have chosen to use the unified process. I chose this
over an agile model as I require more structure than agile gives and I don't feel I will have something to deliver after every iteration
(Some of it will just be knowledge gained). I chose this over waterfall as I find waterfall too directed to the delivery of a software.
% TODO - Insert additional reasoning

% TODO - Insert picture of Unified process

In the inception phase I will undertake research into the current state of accessibility on the web, identify the initial scope and gather
a collection of resources which will support me throughout. The products of this will be:
\begin{itemize}
  \item An understanding of what I want the tool to do, and how much I will build
  \item A GitHub repository with a list of useful links
  \item A Risk Register
  \item A blog which will enable others to watch my progress
\end{itemize}

Following this, the elaboration phase. During this I will continue with my research and use the tools assistive users use to analyse,
assess and improve current documentation. This will give me a solid grounding before I begin writing the tool for assesing such problems,
 and is my first project deliverable.

This elaboration will have two distinct outputs:
\begin{itemize}
  \item Documented examples of good/bad accessibility practices
  \item Software documentation on the tool I will produce (Sequence diagram, use case, user interface design)
\end{itemize}

The construction phase will be when I produce the tool. The tool will consist of two separate products, the first a framework for plugging
in and assessing the accessibility of a web page. The second, a collection of rules which will produce errors, warnings or successes when
assessing the page. This will be the main deliverable.

Finally, the transition phase will involve producing documentation on using, extending and maintaining the project.

% TODO - Reference Capgemini Values
Following Capgemini's Bold value I want to drive this project in a different manner to most academic 'final projects'. Every aspect of the
project will be available on GitHub or the blog from thought through to completion.

\subsection{Project Management}
To ensure the project is delivered on time I use Trello...
% TODO - Insert picture of trello

\subsection{Dissertation Manifesto}
% TODO - discuss purporse, value gained and in evaluation when it was used!!

\subsection{Blog/Project Diary}
% TODO - discuss purporse and value gained

\subsection{Noteworthy entries}
% TODO - discuss Travis CI for report writing
% TODO - discuss Branching model
% TODO - discuss any Standards

\section{Pre-requisite knowledge}
\subsection{Identifying user problems}
\subsection{Understanding IDE analysis methods}
\subsection{Identification of common web frameworks}
\subsection{Identification of common web components}

\section{Software Design}
\subsection{Requirements}
\subsection{Diagrams}
\subsubsection{Sequence Diagram}

\newthought{Lorem ipsum dolor sit amet}

Principally, this chapter should describe the work that was undertaken before
code was written, hardware built, theories worked on, or research studies
executed. It should show how the project proposal was further refined and
clarified, so that the implementation/research execution stage could go
smoothly rather than by trial and error. This part of the report is attempting to
prove that you went through a planning process before embarking on the
deliverable so among others it might include discussion of:

1. For software projects, a requirements analysis, HCI designs, architectural
and use-case diagrams, etc.
2. Any programming languages learnt, any complicated theories or algorithms
that required understanding
3. For research-based projects, the research approach (experimental design,
where applicable), including methods and tools that were used/applied. If the
research method involves the development of prototypic software to test a
concept, briefly describe the design, structure, and creation of this software.
Research methods should be described such that a third party could replicate
the study/experiment to validate the results.

This section should be answering the question “How did I plan to achieve the
deliverable?”
