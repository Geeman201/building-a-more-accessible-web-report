%!TEX root = ../dissertation.tex
\begin{savequote}[75mm]
This is some random quote to start off the chapter.
\qauthor{Firstname lastname}
\end{savequote}

\chapter{Preparation}

\section{Identifyion of Stakeholders}
\section{Way of working}
\subsection{Project Management}
\subsection{Dissertation Manifesto}
\subsection{Blog/Project Diary}

\section{Pre-requisite knowledge}
\subsection{Identifying user problems}
\subsection{Understanding IDE analysis methods}
\subsection{Identification of common web frameworks}
\subsection{Identification of common web components}

\section{Software Design}
\subsection{Requirements}
\subsection{Diagrams}
\subsubsection{Sequence Diagram}

\newthought{Lorem ipsum dolor sit amet}

Principally, this chapter should describe the work that was undertaken before
code was written, hardware built, theories worked on, or research studies
executed. It should show how the project proposal was further refined and
clarified, so that the implementation/research execution stage could go
smoothly rather than by trial and error. This part of the report is attempting to
prove that you went through a planning process before embarking on the
deliverable so among others it might include discussion of:

1. For software projects, a requirements analysis, HCI designs, architectural
and use-case diagrams, etc.
2. Any programming languages learnt, any complicated theories or algorithms
that required understanding
3. For research-based projects, the research approach (experimental design,
where applicable), including methods and tools that were used/applied. If the
research method involves the development of prototypic software to test a
concept, briefly describe the design, structure, and creation of this software.
Research methods should be described such that a third party could replicate
the study/experiment to validate the results.

This section should be answering the question “How did I plan to achieve the
deliverable?”
