%!TEX root = ../dissertation.tex
\begin{savequote}[75mm]
"When we are no longer able to change a situation, we are challenged to
change ourselves"
\qauthor{Viktor E Frankl}
\end{savequote}

\chapter{Background}
% TODO
%  * 20% = https://www.gov.uk/government/uploads/system/uploads/attachment_data/file/600465/family-resources-survey-2015-16.pdf
%  * Reference Disability Discrimintaion Act: http://www.legislation.gov.uk/ukpga/1995/50/contents
%  * Reference JAWS: http://www.freedomscientific.com/Products/Blindness/JAWS
%  * Reference WCAG: https://www.w3.org/TR/WCAG20/
%  *
\newthought{20\% of the UK population reported they have a disability} in
2016. That is
approximately 13.3 million people. In the physical world, companies are by law
bound to ensure that this minority are able to access their services; be it by
leaving enough room to accomodate wheel chairs; or offering large text prints
of their products. In the digital world there are no such laws and thus the web can be a
difficult place for these users to consume services and content.

A range of assistive
tools aim to improve the experience by targeting a selection of
disabilities and offering other means to consume the content. For example,
JAWS targets visually impaired users; reading the content,
labelling actions and offering keyboard shortcuts to navigate. The
problem
with all these tools is that they rely upon Software Engineers to produce
content semantically (using HTML) and add metadata (WCAG) to enable the tools
to better process the content.

The core purpose of this project is to enable the development community to
produce content which is as 'accessible' as possible.

\section{What is accessibility?}
% TODO
%  * Reference Oxford Dictionary
%  * Reference Tim Bernes-Lee: https://www.w3.org/standards/webdesign/accessibility
%  * Reference Adobe: http://www.adobe.com/uk/accessibility/gettingstarted.html
%  * This feels a bit 'bitty' I think it requires some additional quotes and
%    reasoning over the meaning
"Accessibility" is a subjective term which offers many opinionated
definitions. The Oxford English Dictionary defines accessiblity as:
\begin{quote}
"The quality of being able to be reached or entered"
\end{quote}

Focussing more towards the web, Tim Berners-Lee 'Inventor of the web' wrote:
\begin{quote}
"The power of the Web is in its universality. Access by everyone regardless
of disability is an essential aspect"
\end{quote}

Both of these point towards accessibility being the ability to access the
information on offer through the web but fail to demonstrate who is
responsible for enabling it.

Adobe are more specific, they identify that designers and developers
 have a key part to play in enabling users to access content.

\begin{quote}
"How users with disabilities access electronic information and how web content
designers and developers enable web pages to function with assistive devices
used by individuals with disabilities"
\end{quote}

To me, accessibility is enabling users with disabilities to have an 'equally good'
experience as those whom are more able. This begins at the
design level by comparing a range of user experiences to achieve the users
goal through different methods (Sound only, Keyboard only). Then the
responsibility that lies on developers is reduced to writing semantic well formed,
well ordered code.


\section{Business Context \& Requirement}
% TODO
%  * Reference 'public/private' sector company
%  * Reference Agile Cost Of Change http://www.agilemodeling.com/essays/costOfChange.htm
%  * Reference John
Capgemini undertakes a range of projects across both public and private
sector. In the current digital climate most projects involve the
development of some form of web frontend. As a Professional Services supplier it
 is quite common for Software Engineers to be multi-skilled and thus web
 development may not be a "core skill". A common problem is that developers
 who have not had sufficient experience or training in the area of producing
 accessible websites/applications are making simple mistakes in both design and
 implementation which result in a poor experience for users of assistive tools.

 These mistakes are only identified later in the project when
 'assistive tool' testing is performed at which point repetitive and sometimes
 significant rework may be necessary. This is costly and completely
 avoidable.

 This project aims to enable developers with limited experience in web
 development to write cleaner, more semantic code. It is through educating
 developers that the primary goal of offering a better experience to users of
 assistive tools can be achieved. By building a tool that can be used during
 the development process the feedback loop on issues will be much
 shorter and as described in agile 'cost of change' the cost of remediation
 much lower. The tool will be supported by an accessibility guide which can
 be used either to teach or to reference when trying to understand.

John Clifford a Senior Software Engineer at Capgemini has written a short
paragraph about Capgemini's need for a project in this area:
\begin{quote}
 % TODO - Talk to John RE project business benefits paragraph
\end{quote}

% * Myths in the community
%  * Expensive
%  * Ugly
%  * No user will use it

\section{User Requirement}
- To discuss with Harry.


\section{Project Goals}
The core aim of this project is to make an impact in both Capgemini and the
wider web development community. By discovering the communities needs first
the project deliverables will be relevant, useful and fill the described
business need. The deliverables are:
\begin{enumerate}
  \item A tool for software engineers which will bootstrap itself into a web
browser, assess and report possible accessibility issues
  \item A suite of documentation which will collate and demonstrate best
practices. It will show what 'assistive tools' output based upon different
HTML semantics. The current most popular tooling will be used to produce this.
  \item This report which will describe the journey, from researching the
topic, to designing and building the tool.
\end{enumerate}

\section{Project Stakeholders}
% TODO - Note sure about this section - It may need to be binned!
This project has a small number of stakeholders but a wide range of beneficiaries.
\begin{itemize}
  \item Users of assistive technologies. This whole project would be
  impossible and have no impact without them.
  \item My two supervisors — Academic and Company, these are the people whom
  the deliverables will be presented too.
  \item Software Engineers — To improve the experience for the users
  web developers \& designers are those who need to produce a better product.
\end{itemize}

\section{Methodology}
% TODO
%  * Reference Capgemini Values: https://www.uk.capgemini.com/careers/working-at-capgemini/our-values-culture
%  * Reference Sprints - SCRUM: https://www.scrum.org/resources/what-is-a-sprint-in-scrum
%  * Refernece Medium: https://medium.com/
%  * Reference Contiunous Delivery through 'The Lean Startup': https://continuousdelivery.com/
%  * Refernece JVM team "How we work": https://capgemini.github.io/development/how-we-work/#share-knowledge
%  * Reference MIT licence: https://opensource.org/licenses/MIT
%  * Reference Zach Holman "Deployment should be a boring task": https://zachholman.com/posts/deploying-software
%  * Reference Travis CI: https://travis-ci.com/
%  * Reference Github Pages: https://pages.github.com/
%  * Reference Spiral Model:
Capgemini employees are encouraged to follow seven core values. Two of them
which stand out to me are "Boldness" and "Fun" \cite{CapValues}, it is these
two that
drive me to try out new things in the aim of team (and consequently self)
improvement.

Although individual, this project serves as no exception and offers an
opportunity to experiment for the full duration of a project rather than a
few 'sprints'. These 'experiments' will be evaluated at the end of
the project for both myself and others to gain value from. Highlighted below
are a few aspects in which the traditional academic model for a project will
be broken and experimentation will prevail:
 \begin{itemize}
  \item Project Diary - Typically this is done on paper, or captured later in
   the project. 'Medium' a blogging platform will instead be used to
   capture key events and entries. At the end of the project this will be
   supplemented with notes from meetings between myself and my acadamic
   supervisor.
   %TODO - Check with HARRY if the below statement is correct. This is
   %TODO   currently an assumption and may be totally incorrect.
  \item Project Deliverables - The products of an academic project are usually
  completed in isolation or as part of a group and made available only upon
  completion of the project. Experimenting with "Continuous Delivery",
  products of this project will be delivered whilst still undergoing active
  development. This will enable 'faster feedback' from other Software
  Engineers and produce a higher quality product
  \item Open Source Everything - Capgemini consumes and
  produces a lot of open source software! With myself sitting within a team
  that are leading this drive, all code produced will be available under a MIT
  Licence on Github.
  \item Share Everything - The JVM team has its own 'ethos' that sits on top
  of Capgemini's core values one of these is to share knowledge within the
  company and the wider development community. This project will follow that
  ethos and everything will be available online.
  \item Automation - Another of the JVM teams 'values'. Repetitive boring tasks
  should be automated at the earliest possible chance! "Deployment should be a
   repetitive boring task" so this will be one of the first things tackled on
   each deliverable. This will allow its process to be tested and validated
   for every change made thereafter; reducing risk of failure later in the
   project. Travis CI and Github Pages will enable this.
 \end{itemize}

\subsection{Making Better Decisions}
% TODO * reference Agile Manifesto
Coming from a background in Agile projects, the "Agile Manifesto" has enabled
me to make and justify difficult decisions under pressure. For example deciding
 if a time constrained delivery should be moved into or out of the next
 sprint. One of the 'experiments' I wish to undertake is whether having such a
 manifesto makes a difference to a projects direction. With this in mind,
 within the evaluation I will discuss moments this was used and how I felt
 it affected the project. The manifesto for this project will be:
\begin{itemize}
  \item Making an impact over getting a good grade
  \item Hypothesis’ and supporting evidence over guessing and assuming
  \item "Start simple with many organic iterations" over "Upfront design"
  \item Final Project over going to the pub
\end{itemize}

\subsection{A Process}
Capgemini projects always follow a process to ensure a quality
product is delivered and the speed is measurable and quantifiable to
stakeholders and clients.

This project will follow the Unified Process. The unified process gives a
structure to different phases throughout the lifecycle of the project
Inception, Elaboration, Construction and Transition. It enables iterations
to occur within each of these phases but focuses the iteration towards the
goal of the phase you are currently in.

\subsubsection{Why the Unified Process?}

Agile is a poor fit for me as a distant learner with a
full time job. Each iteration requires a small amount of every aspect of the
development lifecycle (analysis, design, implementation and test). This would
make it difficult to understand and report progress to myself and the
stakeholders. Given the projects deliverables are well known from
the outset Agile would only add unneccesary ceremonies.

Using the manifesto above, the sprial model would satisfy "Start simple with
many organic iterations", due to each turn of the spiral gaining user feedback
and then using that feedback to direct new requirements. It's flexible enough
 and well structured for this project. What steered the project towards Unified
 Process over Spiral was the amount of risk
collection and analysis required. Due to being an individual project, most
risks fall into the low probability, low impact category, and with one quater of
each spiral dedicated to risk analysis .

\subsection{High Level Planning}
% * TODO - GANTT CHART of initial plan
** Insert Chart

\section{A Major Issue}
Due to unforseen circumstances ~6 weeks of time was lost; the end
of the elaboration, and most of the construction phase.

\subsection{Change of process}
The decision was made to change from the 'Unified Process' towards a more
prototypical/iterative process; 'Evolutionary Prototyping'. This is a form of
the prototyping model which often suits tight time scales and fast development
lifecycles. The 'Evoloutionary' aspect means rather than the prototype being
thrown away and rebuilt at the end, it will instead be iteratively enhanced
until a final product is produced.

This method enables the code and project to carry some technical debt whilst
still using continuous delivery. There is a higher
risk to the quality of the product but as this is an individual process so I am in
control of that and thus it will be consistent. All technical debt
accumulated and areas for improvement will be documented in the
deliverable/evaluation section respectively.

\subsection{Change of plan}
Due to lost time the plan also had to change. See below for the updated chart:

** Insert Chart

