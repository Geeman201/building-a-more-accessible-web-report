%!TEX root = ../dissertation.tex
\begin{savequote}[75mm]
"When we are no longer able to change a situation, we are challenged to
change ourselves"
\qauthor{Viktor E Frankl}
\end{savequote}

\chapter{Background}
% TODO
%  * 20% = https://www.gov.uk/government/uploads/system/uploads/attachment_data/file/600465/family-resources-survey-2015-16.pdf
%  * Reference Law's required
%  * Reference JAWS
%  * Reference WCAG
%  *
\newthought{20\% of the UK population reported they have a disability} in
2016, that is
approximately 13.3 million people. In the physical world companies are by law
bound to ensure that this minority are able to access their services, be it by
leaving enough room to accomodate wheel chairs, or offering large text prints
of their products. In the digital world there are no such laws and thus the web can be a
difficult place for these users to consume services and content.

A range of assistive
tools aim to improve the experience by targeting a selection of
disabilities and offering other means to consume the content. For example
JAWS targeted at visually impaired users; reading the content,
labelling actions and offering keyboard shortcuts to navigate. The
problem
with all these tools is that they rely upon Software Engineers to produce
content semantically (using HTML) and add metadata (WCAG) to enable the tools
to better process the content.

The core purpose of this project is to enable the development community to
produce content which is as 'accessible' as possible.

\section{What is accessibility?}
% TODO
%  * Reference Oxfrod Dictionary
%  * Reference Tim Bernes-Lee: https://www.w3.org/standards/webdesign/accessibility
%  * Reference Adobe: http://www.adobe.com/uk/accessibility/gettingstarted.html
%  *
"Accessibility" is a subjective term which offers many opinionated
definitions. The Oxford English Dictionary defines accessiblity as:
\begin{quote}
"The quality of being able to be reached or entered"
\end{quote}

Focussing more towards the web, Tim Berners-Lee 'Inventor of the web' wrote:
\begin{quote}
"The power of the Web is in its universality. Access by everyone regardless
of disability is an essential aspect"
\end{quote}

Both of these point towards accessibility being the ability to access the
information on offer through the web but fail to demonstrate who is
responsible for enabling it.

Adobe are more specific, they identify that designers and developers
 have a key part to play in enabling users to access content.

\begin{quote}
"How users with disabilities access electronic information and how web content
designers and developers enable web pages to function with assistive devices
used by individuals with disabilities"
\end{quote}

To me, accessibility is enabling users with disabilities to have an 'equally good'
experience as those whom are more able. This begins at the
design level by comparing a range of user experiences to achieve the users
goal through different methods (Sound only, Keyboard only). Then the
responsibility that lies on developers is reduced to writing semantic well formed,
well ordered code.


\section{Business Context \& Requirement}
% TODO
%  * Reference 'public/private' sector company
%  * Reference John's Quote
Capgemini undertakes a range of projects across both public and private
sector. In the current digital climate most projects involve the
development of some form of web frontend. As a Professional Services supplier it
 is quite common for Software Engineers to be multi-skilled and thus web
 development may not be a "core skill". However, the developer may have
 sufficient skill to produce basic front ends. It is developers whom have
 not had adequate training/experience in producing an accessible experience
 that tend to produce inaccessible websites.
 Often the issues are only identified when performing 'assistive tool
 testing' later in the project. This result is repetitive and sometimes
 significant rework that is costly and completely avoidable.

John Clifford a Senior Software Engineer at Capgemini has written a short
paragraph about Capgemini's need for a project in this area:
\begin{quote}
 % TODO - Talk to John RE project business benefits paragraph
\end{quote}

* Myths in the community
  * Expensive
  * Ugly
  * No user will use it

\section{User Requirement}


\section{Project Goals}
The core aim of this project is to make an impact in both Capgemini and the
wider web development community. By discovering the communities needs first
the project deliverables will be relevant, useful and fill the described
business need. The deliverables are:
\begin{enumerate}
  \item A tool for software engineers which will bootstrap itself into a web
browser, assess and report possible accessibility issues
  \item A suite of documentation which will collate and demonstrate best
practices. It will show what 'assistive tools' output based upon different
HTML semantics. The exmaples will be produced using the current most popular
tooling
  \item This report which will describe the journey, from researching the
topic, to designing and building the tool.
\end{enumerate}

\section{Project Stakeholders}
This project has a small number of stakeholders.
\begin{itemize}
  \item Users of assistive technologies, without them this whole project would be impossible and have no impact.
  \item My two supervisors — Academic and Company, these are the people I will eventually present my deliverables to (as well as the
  \section{Methodology}).
  \item Web development engineers — If I want to improve the experience for the users these are the people that need to produce better code
\end{itemize}

\section{Methodology}
% TODO
%  * Reference Capgemini Values
%  * Reference Sprints - SCRUM
%  * Refernece Medium
%  * Reference Contiunous Delivery through 'The Lean Startup'
%  * Refernece
Capgemini employees are encouraged to follow seven core values. Two of them
which stand out to me are "Boldness" and "Innovation", it is these two that
drive me to try out new things in the aim of team (and consequently self)
improvement.

Although individual, this project serves as no exception and serves as
opportunity to experiment for the full duration of a project rather than just a
few 'sprints'. From an academic perspective I wish to experiment with some of
 the following things:
 \begin{itemize}
  \item Project Diary - Typically this is done on paper, or captured later in
   the project. I want to use 'Medium' a free blogging platform to capture
   individual blog entries
  \item Project Deliverables - The deliverables are usually completed in
  isolation or as part of a group and made available only upon completion of
  the project. Experimenting with Continuous Delivery I am hoping
  that all value will be instantly available and published online. This will
  be done using tools such as Github, Github Pages and Travis CI. More detail
   on this will follow later in the report
 \item Sharing Progress - To ensure
 \end{itemize}

Being a Software Engineer, Project Management is an area in
which I have only received theoretical knowledge rather than industry experience.

An area in which as Software Engineer I have only had
theoretical experience in is project management.

Following the advice in the lean startup for trying new things
 in a structured manner; not to dissimilar to a scientific experiment. The
 process for testing these hypothesese will be as follows:
  \begin{enumerate}
    \item Produce a theory or hypothesis
    \item Design how to experiment
    \item Do the minimum amount of effort to implement the design
    \item Evaluate, to prove or disprove the hypothesis
  \end{enumerate}


drive
this project
in a different
manner
to most academic 'final projects'. Every aspect of the
project will be available on GitHub or the blog from thought through to completion.
\subsection{}
Continuous Delivery


\section{Current Material}




