%!TEX root = ../dissertation.tex
\begin{savequote}[75mm]
Nulla facilisi. In vel sem. Morbi id urna in diam dignissim feugiat. Proin molestie tortor eu velit. Aliquam erat volutpat. Nullam ultrices, diam tempus vulputate egestas, eros pede varius leo.
\qauthor{Quoteauthor Lastname}
\end{savequote}

\chapter{Background}
% TODO
%  * 20% = https://www.gov.uk/government/uploads/system/uploads/attachment_data/file/600465/family-resources-survey-2015-16.pdf
%  * Reference Law's required
%  * Reference JAWS
%  * Reference WCAG
%  *
\newthought{20\% of the UK population reported they have a disability} in
2016, that is
approximately 13.3 million people. In the physical world companies are by law
bound to ensure that this minority are able to access their services, be it by
leaving enough room to accomodate wheel chairs, or offering large text prints
of their products. In the digital world there are no such laws and thus the web can be a
difficult place for these users to consume services and content.

A range of assistive
tools aim to improve the experience by targeting a selection of
disabilities and offering other means to consume the content. For example
JAWS is aimed at users whom are visually impaired; reading the content,
labelling actions and offering additional keyboard shortcuts to navigate. The
problem
with all these tools is that they rely upon Software Engineers to produce
content semantically (using HTML) and add metadata (WCAG) to enable the tools
to better process the content.

The core purpose of this project is to enable the development community to
produce content which is as 'accessible' as possible.

\subsubsection{What is accessibility?}
% TODO
%  * Reference Tim Bernes-Lee: https://www.w3.org/standards/webdesign/accessibility
%  * Reference
"Accessibility" is a subjective term which offers many opinionated
definitions.

Tim Berners-Lee 'Inventor of the web' wrote:
\begin{quote}
"The power of the Web is in its universality. Access by everyone regardless
of disability is an essential aspect"
\end{quote}

This is a rather broad statement but suggests accessibility is merely the
ability to access the information on offer through the web.

XYZ wrote:
\begin{quote}

\end{quote}


\subsubsection{User Requirement}
\subsubsection{Business Context}
* Myths in the community
  * Expensive
  * Ugly
  * No user will use it
\subsubsection{Business Requirement}
\subsubsection{Current Material}

\subsubsection{Project Goals}
This project will produce three distinct deliverables. The first, a tool for software engineers which will bootstrap itself into a web
browser, assess and then report possible accessibility issues. The second, a suite of documentation which will demonstrate the issues, and
show engineers how to remediate the problems - these will be produced using the current most popular tooling. The third, this report,
this will describe the journey, from researching the topic, to designing and building the tool.

\subsubsection{Methodology}
