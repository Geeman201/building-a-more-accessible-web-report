%!TEX root = ../dissertation.tex
\begin{savequote}[75mm]
Nulla facilisi. In vel sem. Morbi id urna in diam dignissim feugiat. Proin molestie tortor eu velit. Aliquam erat volutpat. Nullam ultrices, diam tempus vulputate egestas, eros pede varius leo.
\qauthor{Quoteauthor Lastname}
\end{savequote}

\chapter{Background}
% TODO
%  * 20% = https://www.gov.uk/government/uploads/system/uploads/attachment_data/file/600465/family-resources-survey-2015-16.pdf
%  * Reference Law's required
%  * Reference JAWS
%  * Reference WCAG
%  *
\newthought{20\% of the UK population reported they have a disability} in
2016, that is
approximately 13.3 million people. In the physical world companies are by law
bound to ensure that this minority are able to access their services, be it by
leaving enough room to accomodate wheel chairs, or offering large text prints
of their products. In the digital world there are no such laws and thus the web can be a
difficult place for these users to consume services and content.

A range of assistive
tools aim to improve the experience by targeting a selection of
disabilities and offering other means to consume the content. For example
JAWS targeted at visually impaired users; reading the content,
labelling actions and offering keyboard shortcuts to navigate. The
problem
with all these tools is that they rely upon Software Engineers to produce
content semantically (using HTML) and add metadata (WCAG) to enable the tools
to better process the content.

The core purpose of this project is to enable the development community to
produce content which is as 'accessible' as possible.

\subsubsection{What is accessibility?}
% TODO
%  * Reference Oxfrod Dictionary
%  * Reference Tim Bernes-Lee: https://www.w3.org/standards/webdesign/accessibility
%  * Reference Adobe: http://www.adobe.com/uk/accessibility/gettingstarted.html
%  *
"Accessibility" is a subjective term which offers many opinionated
definitions. The Oxford English Dictionary defines accessiblity as:
\begin{quote}
"The quality of being able to be reached or entered"
\end{quote}

Focussing more towards the web, Tim Berners-Lee 'Inventor of the web' wrote:
\begin{quote}
"The power of the Web is in its universality. Access by everyone regardless
of disability is an essential aspect"
\end{quote}

Both of these point towards accessibility being the ability to access the
information on offer through the web but fail to demonstrate who is
responsible for enabling it.

Adobe are more specific, they identify that designers and developers
 have a key part to play in enabling users to access content.

\begin{quote}
"How users with disabilities access electronic information and how web content
designers and developers enable web pages to function with assistive devices
used by individuals with disabilities"
\end{quote}

To me, accessibility is enabling users with disabilities to have an 'equally good'
experience as those whom are more able. This begins at the
design level by comparing a range of user experiences to achieve the users
goal through different methods (Sound only, Keyboard only). Then the
responsibility that lies on developers is reduced to writing semantic well formed,
well ordered code.


\subsubsection{Business Context & Requirement}
% TODO
%  * Reference 'public/private' sector company
%  * Reference John's Quote
Capgemini undertakes a range of projects across both public and private
sector. In the current digital climate most projects involve the
development of some form of web frontend. As a Professional Services supplier it
 is quite common for Software Engineers to be multi-skilled and thus web
 development may not be a "core skill", but a developer may have sufficient
 skill to produce basic front ends. It is these developers whom have
 not had adequate training in producing an accessible experience. The result
 is issues are only identified when performing 'assistive tool testing'
 later in the project. This results in repetitive and sometimes significant
 rework that is costly and completely avoidable.

John Clifford a Senior Software Engineer at Capgemini has written a short
paragraph about Capgemini's the need for a project in this area:
\begin{quote}
 % TODO - Talk to John RE project business benefits paragraph
\end{quote}

* Myths in the community
  * Expensive
  * Ugly
  * No user will use it

\subsubsection{User Requirement}

\subsubsection{Current Material}


\subsubsection{Project Goals}
This project will produce three distinct deliverables. The first, a tool for software engineers which will bootstrap itself into a web
browser, assess and then report possible accessibility issues. The second, a suite of documentation which will demonstrate the issues, and
show engineers how to remediate the problems - these will be produced using the current most popular tooling. The third, this report,
this will describe the journey, from researching the topic, to designing and building the tool.

\subsubsection{Methodology}
